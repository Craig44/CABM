\section{Model overview\label{sec:overview}}\index{Model overview}

\subsection{Introduction}

\IBM\ is a generalised individual based model. 

\IBM\ is run from the console window in Microsoft Windows or from a terminal window in Linux. \IBM\ gets its information from input data files, the main one of which is the \emph{\config}. Commands and subcommands in the \config\ are used to define the model structure, provide observations, define parameters, and define the outputs (reports) for \IBM. Command line switches tell \IBM\ the run mode and where to direct its output. See Section~\ref{sec:running} for details.

We define the model in terms of the \emph{state}\index{Model ! state}\index{State}. The state consists of two parts, the \emph{partition}\index{Model ! partition}\index{Partition}, and any \emph{derived quantities}\index{Model ! derived quantities}\index{Derived quantities}. The state will typically change in each \emph{time-step}\index{Model ! time-steps}\index{time-steps} of every year, depending on the \emph{processes}\index{Model ! processes}\index{Processes} defined for those time-steps in the model. 

\subsection{\I{The population section}}


\subsection{\I{The estimation section}}

\subsection{\I{The observation section}}

\subsection{\I{The report section}}
