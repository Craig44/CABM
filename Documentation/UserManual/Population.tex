\section{\I{The population section}\label{sec:population-section}}

\subsection{Introduction}
The population section\index{Population section} specifies the model of the population dynamics. It describes the model structure (population structure), defines the population processes (e.g., recruitment, migration, and mortality), the selectivities, and associated parameters.

The population section consists of several components, including:
\begin{itemize}
  \item The population structure;
  \item Model initialisation (i.e., the state of the partition at the start of the first year)\index{Initialisation}\index{Model ! initialisation};
  \item The years over which the model runs (i.e., the start and end years of the model)
  \item The annual cycle (time-steps and processes that are applied in each time-step)\index{Annual cycle};
  \item The specification and parameters of the population processes (i.e., processes that add, remove individuals to or from the partition, or shift numbers between ages and categories in the partition);
  \item Selectivities;
  \item The spatial domain via defining layers
  \item Parameter values and their definitions; and
  \item Derived IBM, required as parameters for some processes (e.g. mature biomass to resolve any density dependent processes, such as the spawner-recruit relationship in a recruitment process).
\end{itemize}

\subsection{\I{Population structure}}\label{sub:sec:pop_sec}
The basic structure of the population section of a \IBM\ model is defined in terms of an annual cycle, time steps, states, and transitions.

The annual cycle defines what processes happen in each model year, and in what sequence. \IBM\ runs on an annual cycle rather than, for example, a 6-monthly cycle.

Each year is split into one or more time steps, with at least one process occurring in each time step. Each time step can be thought of as representing a particular part of the calendar year, or time steps can be treated as an abstract sequence of events. In every time step, there exists a mortality block: a group of consecutive mortality-based processes, where individuals are removed from the partition (see Section~\ref{sec:mortality_block}).

The state is the current status of the population at any given time. The state can change one or more times in each time step of every year. The state object must contain sufficient information to figure out how the underlying population changes over time (given a model and a complete set of parameters).

The state can undergo a number of possible changes, called transitions. Transitions are accomplished by processes, including: recruitment, natural mortality, anthropogenic mortality, ageing, migration, tagging events, and maturation. 

The division of the year into an arbitrary number of time steps allows the user to specify the exact order in which processes and observations occur throughout the year. The user needs to specify the time step in which each process occurs. If more than one process occurs in the same time step, the order in which to apply each process is specified in the \command{time\_step} block.

The key element of the state is the spatial world view, which holds all the entities.

An example, to specify a model with 2 categories (male and female) with ages 1-20 (with the last age a plus group) and an age-length relationship defined with the label \texttt{male\_growth} and \texttt{female\_growth}, then the \texttt{@model} block is specified as:
{\small{\begin{verbatim}
		@model
		start_year
		final_year
		min_age 1
		max_age 20
		age_plus_group True
		initialisation_phases iphase
		time_steps step1 step2 step3
\end{verbatim}}}

\subsection{\I{The state object and the partition}}



\subsection{\I{Time sequences}}

The time sequence of the model is defined in the following parts;
\begin{itemize}
  \item \I{Annual cycle}
  \item \I{Mortality blocks}
  \item \I{Initialisation}
  \item \I{Model run years}\textsl{}
\end{itemize}

\subsubsection{\I{Annual cycle}}
The annual cycle is implemented as a set of processes that occur, in a user-defined order, within each year. Time-steps are used to break the annual cycle into separate components, and allow observations to be associated with different time periods and processes. Any number of processes can occur within each time-step, in any order (although there are limitations around mortality based processes - see Section~\ref{sec:mortality_block}) and can occur multiple times within each time-step. Note that time-steps are not implemented during the initialisation phases (effectively, there is only one time-step), and that the annual cycle in the initialisation phases can, optionally, be different from that which is applied during the model years.

\subsubsection{\I{Mortality blocks}}\label{sec:mortality_block}

For every time step in an annual cycle there is an associated \emph{mortality block}. Mortality blocks are a key concept in \IBM.

Mortality blocks are used to define the `point' in the model time sequence when observations (see Section~\ref{sec:observation-section}) are evaluated, and derived quantities (see Section~\ref{sec:derived-quantities}) are calculated.

A mortality block is defined as a consecutive sequence of mortality processes within a time step. The processes that are mortality processes are all pre-defined in \IBM, and cannot be modified. These mortality processes are described in subsection~\ref{sec:mortality}. 

\IBM\ requires that each time step has exactly one mortality block. To achieve this, either all the mortality processes in a time step must be sequential (i.e., there can not be a non-mortality process between any two mortality processes within any one time step); or if no mortality processes occur in a time step then the mortality block is defined to occur at the end of the time step. 

\IBM\ will error out if more than one mortality block occurs in a single time step. 


\subsubsection{\I{Initialisation}}\label{subsec:initialisation}
Initialisation is the process of determining the world's state just before \subcommand{start\_year}, whether it be equilibrium/steady state or some other initial state for the model (e.g exploited), prior to the start year of the model. This can be computationally expensive if a plus group is present in the partition.

Currently users can only initialise the partition via an iterative process. \IBM\ does a few tricks to help speed up the initialisation. The first thing \IBM\ does is gets the parameter \subcommand{number\_of\_agents} and spits that number of agents uniformly, over the spatial domain, alternatively the user could supply a layer \subcommand{layer\_label} of proportions to seed the initial spatial distribution. This layer should sum to one so that the model initially seeds \subcommand{number\_of\_agents}. When the \IBM\ seeds the initial number of agents it also randomly assigns the agents an age based on an exponential distribution where the parameter $\lambda$ of the exponential distribution is set by the command on the \command{model}, \subcommand{natural\_mortality\_process\_label}. We suggest setting this command to the assumed natural mortality of the model. To see what age structure this would look like you can quickly use \R\ to visualise. by running the following code in an \R\ terminal you could see.

\begin{lstlisting}
Z_param = 0.2;
agents_per_cell = 1000;
hist(rexp(agents_per_cell,Z_param), breaks = 30, xlab = "age", ylab = "frequency", main = "Initial age structure in each cell")
\end{lstlisting}

Once \IBM\ has seeded the agents, it iterates over the annual cycle to change an approximated initialisation state to one that is more like what would occur for your annual cycle, this is controlled by the \subcommand{years} command in the \command{initialisation} block. The number of iterations in the iterative initialisation can effect the model output, and these should be chosen to be large enough to allow the population state to fully converge. We recommend that a period of about two generations to ensure convergence. \

Hence, for an iterative initialisation you need to define:
\begin{itemize}
  \item The initialisation phases,
  \item The number of years in each phase, and
  \item the natural mortality process
  \item the growth process
\end{itemize}

Because the initialisation phase is responsible for seeding the initial agents, users must specify processes that the initialisation phase can seed parameters to agents. To see how parameters are set for each individual agent, users should see the individual processes in this Section~\ref{sec:process}.

An example of the syntax to implement this would be,
{\small{\begin{verbatim}
@model
...
initialisation_phases Iterative_initialisation

@initialisation_phase Iterative_initialisation
type iterative
years 50
lambda 0.0001
convergence_years 20 40
layer_label Base
growth_process_label von_bert
natural_mortality_process_label natural_mort
\end{verbatim}}}

\subsection{\I{Processes}}\label{sec:process}
Processes are a set of classes that get access to agents parameters and either modifies them (growth), moves them (to anther cell), adds new agents (recruitment) or removes them (mortality). Some users may not think having an analagous to population processes fits in the Individual based model framework, but it is structured this way for convenience and code maintenance more than anything. The other thing to note is that most proceses are dictated by agent specific parameters such as natural mortality and growth etc.


\subsubsection{\I{Ageing}}
Ageing is an implicit process in the model, Each agent that is created or recruited gets assigned an birth year. This means that when ever we want to ask for the agent we just calculate \subcommand{current\_year - birth\_year}, thus there is no explicit ageing process. Note that we do return the \subcommand{max\_age} of the model, if the agent is older than that age (so we only work with the truncated age distribution).


This means every fish automatically ages by one at the end of the year, or you could think of ageing a fish at the very beginning of the year (tomayto, tomahto), just be aware that you don't  have control over that, but if you want to account for growth in between annual increments that is possible via time step proportion increments, see growth processes for more information.

\subsubsection{\I{Recruitment}}

\paragraph{\I{Constant recruitment}}\label{subsubsec:constant-recruitment}

\paragraph{\I{Beverton-Holt recruitment}}\label{subsubsec:bev-holt-recruitment}


\subsubsection{\I{Mortality}\label{sec:mortality}}
Mortality processes remove individuals from the model domain (and also from memory).

\paragraph{Constant mortality rate}

\paragraph{Event-Biomass}
This is a removal event such as fishing, where we iterate over a range of cells and remove agents based on some selectivity or vulnerability.



\subsection{\I{Derived Quantities}\label{sec:derived-quantities}}
Derived quantities surround a mortality block and so the value of the derived quantity is calculated twice, once before the mortality block, and once after. The final value is an interpolation based on how much mortality the user wants to into account when calculating derived quantities.
\paragraph*{Biomass}
\paragraph*{Mature Biomass}
\paragraph*{Abundance}

\subsection{\I{Growth}\label{sec:age-at-age}}
Growth currently means updating length and weight for an agent, if the Von Bertalanffy formula currently is used (currently the default in agent class for initialising length) it follows the following formula

\begin{equation}\label{VB}
	\Delta L_i = (L_{i,\infty} - L_i)(1 - e^-{k_i})
\end{equation}


Where $i$ indexes each agents own length and growth parameters. If the basic length weight formulation is used, then an agents weight follows the following formula.

\begin{equation}\label{mean_weight}
\bar{w_i} = a_iL_i^{b_i}
\end{equation}


When an agent is created (either in initialisation or through a recruitment process) or moves cell, it gets assigned new growth parameters that relate to that cell. This allows for spatial growth, one could hypothesis that environment may explain growth and so different environments over cells will cause different rates of growth. For the growth process you must specify either a spatial layer for mean values of each growth parameter or a single value (if growth doesn't change through space), a distribution and Coefficient of variation (CV). This randomly generates an agent a parameter from that distribution, an example of the syntax in a four area model,

{\small{\begin{verbatim}
		@model
		...
		initialisation_phases Iterative_initialisation
		
		@initialisation_phase Iterative_initialisation
		type iterative
		years 50
		lambda 0.0001
		convergence_years 20 40
		layer_label Base
		growth_process_label von_bert
		natural_mortality_process_label natural_mort
\end{verbatim}}}



\subsection{\I{Selectivities}\label{sec:selectivities}}
A selectivity is a function that can have a different value for each age class. Selectivities are used throughout \IBM\ to interpret observations (Section \ref{sec:estimation-section}) or to modify the effects of processes on each age class (Section \ref{sec:population-section}). \IBM\ implements a number of different parametric forms, including logistic, knife edge, and double normal selectivities. Selectivities are defined in there own command block (\command{selectivity}), where the unique label is used by observations or processes to identify which selectivity to apply.

Selectivities are indexed by age, with indices from \argument{min\_age} to \argument{max\_age}. For example, for a logistic age-based selectivity with $50\%$ selected at age $5$ and $95\%$ selected at age $7$, would be defined by the \subcommand{type}=\argument{logistic} with parameters $a_{50}=5$ and $a_{to95}=(7-5)=2$. The value of the selectivity at age $x=7$ is $0.95$, and the value at age $x=3$ is $0.05$. Note, while selectivities can be length based, use with caution as more testing is needed for this functionality.

The function values for some choices of parameters, for some selectivities, can result in a computer numeric overflow error (i.e., the number calculated from parameter values is either too large or too small to be represented in computer memory). \IBM\ implements range checks on some parameters to test for a possible numeric overflow error before attempting to calculate function values. For example, the logistic selectivity is implemented such that if $(a_{50}-x)/a_{to95} > 5$ then the value of the selectivity at $x=0$, i.e., for $a_{50}=5$, $a_{to95}=0.1$, then the value of the selectivity at $x=1$, without range checking would be $7.1 \times 10^{-52}$. With range checking, that value is $0$ (as $(a_{50}-x)/a_{to95}=40 > 5$).

The available selectivities are;

\begin{itemize}
  \item Constant
  \item Knife-edge
  \item All values
  \item All values bounded
  \item Increasing
  \item Logistic
  \item Inverse logistic
  \item Logistic producing
  \item Double normal
  \item Double exponential
% \item Cubic spline (Not yet implemented)
\end{itemize}

The available selectivities are described below.

\subsubsection[Constant]{{constant}}

\begin{equation}
f(x)=C
\end{equation}

The constant selectivity has the estimable parameter C. 

\subsubsection[Knife-edge]{\argument{knife\_edge}}
\begin{equation}
f(x)= \begin{cases}
  0, & \text{if $x < E$} \\
  \alpha, & \text{if $x \ge E$}\\ 
  \end{cases} 
\end{equation}

The knife-edge ogive has the estimable parameter E and a scaling parameter $\alpha$, where the default value of $\alpha = 1$.

\subsubsection[All-values]{\argument{all\_values}}\index{Selectivities!All-values}

\begin{equation}
f(x)=V_x
\end{equation}

The all-values selectivity has estimable parameters $V_{low}$, $V_{low+1}$ \ldots $V_{high}$. Here, you need to provide the selectivity value for each age class.

\subsubsection[All-values-bounded]{\argument{all\_values\_bounded}}\index{Selectivities!All-values-bounded}

\begin{equation}
f(x)=\begin{cases}
		 0, & \text{if $x < L$} \\
		 V_x, & \text{if $L \le x \le H$} \\
		 V_H, & \text{if $x > H$}
  \end{cases}
\end{equation}

The all-values-bounded selectivity has non-estimable parameters L and H. The estimable parameters are $V_L$, $V_{L+1}$ \ldots $V_H$. Here, you need to provide an selectivity value for each age class from $L \ldots H$.

\subsubsection[Increasing]{\argument{increasing}}\index{Selectivities!Increasing}

\begin{equation} 
f(x)=\begin{cases}
	  0, & \text{if $x < L$} \\
	  f(x-1)+ \pi_x(\alpha-f(x-1)), & \text{if $L \le x \le H$} \\
	  f(\alpha), & \text{if $x \ge H$} \\  
  \end{cases}
\end{equation}

The increasing ogive has non-estimable parameters $L$ and $H$. The estimable parameters are $\pi_L$, $\pi_{L+1}$ \ldots $\pi_H$ (but if these are estimated, they should always be constrained to be between 0 and 1). $\alpha$ is a scaling parameter, with default value of $\alpha = 1$. Note that the increasing ogive is similar to the all-values-bounded ogive, but is constrained to be non-decreasing.

\subsubsection[Logistic]{\argument{logistic}}\index{Selectivities!Logistic}

\begin{equation}
  f(x) = \alpha / [1+19^{(a_{50}-x)/a_{to95}}]
\end{equation}
 
The logistic selectivity has estimable parameters $a_{50}$ and $a_{to95}$. $\alpha$ is a scaling parameter, with default value of $\alpha = 1$. The logistic selectivity takes values $0.5 \alpha$ at $x=a_{50}$ and $0.95 \alpha$ at $x=a_{50}+a_{to95}$. 

\subsubsection[Inverse logistic]{\argument{inverse\_logistic}}\index{Selectivities!Inverse-logistic}

\begin{equation}
  f(x) = \alpha - \alpha / [1+19^{(a_{50}-x)/a_{to95}}]
\end{equation}
 
The inverse logistic selectivity has estimable parameters $a_{50}$ and $a_{to95}$. $\alpha$ is a scaling parameter, with default value of $\alpha = 1$. The logistic selectivity takes values $0.5 \alpha$ at $x=a_{50}$ and $0.95 \alpha$ at $x=a_{50}-a_{to95}$. 

\subsubsection[Logistic producing]{\argument{logistic\_producing}}\index{Selectivities!Logistic-producing}

\begin{equation} 
f(x)=\begin{cases}
	  0, & \text{if $x < L$} \\
	  \lambda(L), & \text{if $x=L$} \\
	  \left( \lambda(x)-\lambda(x-1) \right) / \left( 1-\lambda(x-1) \right), & \text{if $L < x < H$} \\
	  1, & \text{if $x \ge H$} \\  
  \end{cases}
\end{equation}

The logistic-producing selectivity has the non-estimable parameters $L$ and $H$, and has estimable parameters $a_{50}$ and $a_{to95}$. $\alpha$ is a scaling parameter, with default value of $\alpha = 1$. For category transitions, $f(x)$ represents the proportion moving, not the proportion that have moved. This selectivity was designed for use in an age-based model to model maturity. In such a model, a logistic-producing maturation selectivity will (in the absence of other influences) make the proportions mature follow a logistic curve with parameters $a_{50}$, $a_{to95}$.

\subsubsection[Double-normal]{\argument{double\_normal}}\index{Selectivities!Double-normal}

\begin{equation}
  f(x) = \begin{cases}
    \alpha 2^{-[(x- \mu)/\sigma_L ]^2}, & \text{if $x \leq \mu$} \\
    \alpha 2^{-[(x- \mu)/\sigma_R ]^2}, & \text{if $x \ge \mu$}\\
  \end{cases}
\end{equation} 

The double-normal selectivity has estimable parameters $a_1$, $s_L$, and $s_R$. $\alpha$ is a scaling parameter, with default value of $\alpha = 1$. It has values $\alpha$ at $x=a_1$, and $0.5 \alpha$ at $x=a_1-s_L$ and $x=a_1+s_R$. 

\subsubsection[Double-exponential]{\argument{double\_exponential}}\index{Selectivities!Double-exponential}

\begin{equation} 
f(x)=\begin{cases}
	  \alpha y_0(y_1 / y_0)^{(x-x_0)/(x_1-x_0)}, & \text{if $x \le x_0$} \\
	  \alpha y_0(y_2 / y_0)^{(x-x_0)/(x_2-x_0)}, & \text{if $x > x_0$} \\
  \end{cases}
\end{equation}

The double-exponential selectivity has non-estimable parameters $x_1$ and $x_2$, and estimable parameters $x_0$, $y_0$, $y_1$, and $y_2$.  $\alpha$ is a scaling parameter, with default value of $\alpha = 1$. It can be `U-shaped'. Bounds for $x_0$ must be such that $x_1 < x_0 < x_2$. With $\alpha=1$, the selectivity passes through the points $(x_1, y)$, $(x_0, y_0)$, and $(x_2, y_2)$. If both $y_1$ and $y_2$ are greater than $y_0$ the selectivity is `U-shaped' with minimum at $(x_0, y_0)$.

%\subsubsection[Spline]{\argument{spline}}\index{Selectivities!Spline}
%
%The spline selectivity implements a cubic spline that has non-estimable knots, and an estimable value for each knot. The cubic spline is either (i) a natural splines where the second derivatives are set to 0 at the boundaries, i.e., the values at the boundaries are horizontal, (ii) a spline with a fixed first derivative at the boundaries (linear, but not necessarily horizontal) and (iii) spline which turns into a parabola at the boundaries. 
%


Selectivities \subcommand{all\_values} and \subcommand{all\_values\_bounded} can be addressed in additional priors using the following syntax,

{\small{\begin{verbatim}
		@selectivity maturity
		type all_values
		v 0.001 0.1 0.2 0.3 0.4 0.3 0.2 0.1
		
		## encourage ages 3-8 to be smooth.
		@additional_prior smooth_maturity
		type vector_smooth
		parameter selectivity[maturity].values{3:8}
		
		\end{verbatim}}}
	
	
\subsection{\I{Tips setting up configuration files}\label{sec:tips}}
\IBM\ can take a while if you have a complex spatial model and complex life history. So there are some tips and tricks that I have come found can be useful when setting up configuration files.

\subsubsection*{The initialisation Phase}
The first thing you wont to nail down before getting gun hoe on the configuration and complex dynamics you want to get a model that reaches an appropriate initialisation state \emph{quickly!}. So my suggestion is to not worry about fishing in the first instance, set the \subcommand{start\_year} and \subcommand{final\_year} one year apart and play with the initialisation phase commands to see how to efficiently reach a desired initial state. The parameters I am talking about are the subcommands \subcommand{years} and \subcommand{layer\_label}. The \subcommand{years} parameter sets how many annual cycles to run through to set your initialisation state, you want this as small as possible. So my advice is run the model with a range of \subcommand{years} say 5, 10, 20, 30, 40, 50 and 100 to find out how sensitive your initial state is to this. You find that you have to make a compromise between speed and accuracy, unfortunately life is full of such compromises.

Along with the \subcommand{years} parameter you can also set the spatial distribution of the initial seeding of agents through the \subcommand{layer\_label}, if you have a spatial model with movement this would be an interesting one to play with. So pretty much my advice is tweak these initialisation parameters as much as you can before you move onto fishing and generate observations, it will be well worth your while if you treasure time.

\subsubsection*{The number of agents in the system} 
This one should be obvious and we suggest that you play with the initialisation parameter \subcommand{number\_of\_agents} to see how sensitive model outputs are to this parameter. The less agents in the system the faster your model run, however the less agents the more coarse you model becomes because then an entity all of a sudden represents 1000 entities, so once again we return to a compromise.

\subsubsection*{The number of threads available} 
\IBM\ is threaded using the OpenMP C++ library, by default \IBM\ will set the number threads available equal to number available on your machine minus 1. You can set this through the \command{model} parameter \subcommand{max\_threads\_to\_use}, we suggest you set this to the number of spatial cells in the system, there might be more overheads with making more threads available than the model will ever use (disclaimer I haven't tested this). 



