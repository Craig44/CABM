\section{\I{The report section}\label{sec:report-section}}\index{Reports}\index{Reports section}
The report section specifies the printouts and other outputs from the model. \IBM\ does \textbf{not} produce any output unless requested by a valid \command{report} block. This important to note, as model runs can take a few minutes to run it will be in vain if there are no reports.

Reports from \IBM\ can be defined to print partition and states objects at a particular point in time, observation summaries, estimated parameters and objective function values. See below for a more extensive list of report types, and an example of an observation report.


\subsubsection{Initialisation Partition}
A very useful report for looking at the initialisation partition. This report prints too instances of the partition. The first is right after we do an exponential approximation in each cell, and the second is after we have run the initial annual cycle for \subcommand{years} and are about to enter the actual annual cycle.

{\small{\begin{verbatim}
		@report init
		type initialisation_partition
\end{verbatim}}}


\subsubsection{Age Frequency}
There are two reports for printing the age frequency, you can print it for the all cells the total age frequency at the end of a time step using the \subcommand{type} \subcommand{world\_age\_frequency}. Or you could choice to print the age frequency for each cell using the \subcommand{type} \subcommand{age\_frequency\_by\_cell}. 
{\small{\begin{verbatim}
# print age frequency by cell
@report age_freq
type age_frequency_by_cell
#years 1990:2018 ## defaults to model years
time_step Annual
\end{verbatim}}}

{\small{\begin{verbatim}
# print age frequency for all cells
@report world_age_freq
type world_age_frequency
#years 1990:2018 ## defaults to model years
time_step Annual
\end{verbatim}}}



\subsubsection{Observation}
This prints a summary of an observation and simulated values, for each year and cell.

{\small{\begin{verbatim}
@observation fishery_age
type process_removals_by_age
simulation_likelihood multinomial
process_label fishing
years 2000
min_age 0
max_age 28
#plus_group true
ageing_error None
table error_values
2000 10000
end_table
cell_layer cells
cells r1-c1	
		
@report fisher_age_freq
type observation
observation fishery_age
\end{verbatim}}}


\subsubsection{Model Attributes}
This pretty much just prints the global scalar used to convert numbers and biomass from agent space to population level.
{\small{\begin{verbatim}
@report model_attributes
type model_attributes
\end{verbatim}}}


\subsubsection{Derived Quantities}
This report will print all the derived quantities for all years in the model.
{\small{\begin{verbatim}
@report derived_quants
type derived_quantity
\end{verbatim}}}

\subsubsection{Process}
This report will print all parameters that were set for a specific process and any auxiliary information that you want from the specific process. Many of the processes will print extra information such as some of the mortality processes that are responsible for tag recaptures.
{\small{\begin{verbatim}
@report M_report
type process
process natural_mort
\end{verbatim}}}


\subsubsection{Numeric Layer}
This report will print out the model derived numeric layers such Abundance and biomass. Useful for showing the spatial distribution of all the agents over time.
{\small{\begin{verbatim}
@report biomass_by_cell
type numeric_layer
layer_label biomass_by_cell
years 1990:2018
time_step Summer
\end{verbatim}}}


\subsubsection{Time varying}
This report will print out all the time varying parameters and the values assumed in each year
{\small{\begin{verbatim}
@report time_varying_parameters
type time)varying
\end{verbatim}}}


\subsubsection{Summarise Agents}
This report will randomly print all the attributes from a user defined number of agents that can be viewed at the end of each time step. Useful looking at length weight by area, tagged fish etc.
{\small{\begin{verbatim}
@report agents
type summarise_agents
years 1990:2018
time_step Summer
number_of_individuals 1000
\end{verbatim}}}

