\section{Quick reference\label{sec:quick-reference}}
\defComLab{derivedquantity}{Define an object of type \emph{derivedquantity}}\par
\defSub{label} {Label of the derived quantity}
\defSub{type} {Type of derived quantity}
\defSub{time\_step} {The time step in which to calculate the derived quantity after}
\defSub{proportion\_through\_mortality\_block} {Proportion through the mortality block of the time step when calculated}
\par\textbf{\commandlabsubarg{derived\_quantity}{type}{abundance}}\par
\defSub{selectivity} {A label for the selectivity}
\defSub{layer\_label} {A label for the layer that indicates which cells to calculate abundnace in over}
\par\textbf{\commandlabsubarg{derived\_quantity}{type}{biomass}}\par
\defSub{selectivity} {A label for the selectivity}
\defSub{biomass\_layer\_label} {A label for the layer that indicates which cells to calculate biomass over}
\par\textbf{\commandlabsubarg{derived\_quantity}{type}{biomass\_by\_cell}}\par
\par\textbf{\commandlabsubarg{derived\_quantity}{type}{mature\_biomass}}\par
\defSub{biomass\_layer\_label} {A label for the layer that indicates which cells to calculate biomass over}
\defComLab{initialisationphase}{Define an object of type \emph{initialisationphase}}\par\par
\defSub{label} {The label of the initialisation phase}
\defSub{type} {The type of initialisation}
\par\textbf{\commandlabsubarg{initialisation\_phase}{type}{iterative}}\par
\defSub{years} {The number of iterations (years) over which to execute this initialisation phase}
\defSub{initial\_number\_of\_agents} {The number of agents to initially seed in the partition}
\defSub{layer\_label} {The label of a layer that you want to seed a distribution by.}
\defSub{recruitment\_layer\_label} {The label of a layer has a recruitment process label in each cell to see how to set scalars}
\defSub{initialisation\_mortality\_rate} {The instaneous mortality rate to use to approximate a crude initial age-structure}
\defComLab{likelihood}{Define an object of type \emph{likelihood}}\par\par
\par\textbf{\commandlabsubarg{likelihood}{type}{binomial}}\par
\par\textbf{\commandlabsubarg{likelihood}{type}{binomial\_approx}}\par
\par\textbf{\commandlabsubarg{likelihood}{type}{dirichlet}}\par
\par\textbf{\commandlabsubarg{likelihood}{type}{log\_normal}}\par
\par\textbf{\commandlabsubarg{likelihood}{type}{log\_normal\_with\_q}}\par
\par\textbf{\commandlabsubarg{likelihood}{type}{logistic\_normal}}\par
\defSub{label} {Label for Logisitic Normal Likelihood}
\defSub{type} {Type of likelihood}
\defSub{rho} {The auto-correlation parameter $\rho$}
\defSub{sigma} {Sigma parameter in the likelihood}
\defSub{arma} {Defines if two rho parameters supplied then covar is assumed to have the correlation matrix of an ARMA(1,1) process}
\defSub{bin\_labels} {If no covariance matrix parameter then list a vector of bin labels that the covariance matrix will be built for, can be ages or lengths.}
\defSub{sexed} {Will the observation be split by sex?}
\defSub{robust} {Robustification term for zero observations}
\defSub{seperate\_by\_sex} {If data is sexed, should the covariance matrix be seperated by sex?}
\defSub{sex\_lag} {if T and data are sexed, then the AR(n) correlation structure ignores sex and sets lag = |i-j|+1, where i and j index the age or length classes in the data.  Ignored if data are not sexed.}
\par\textbf{\commandlabsubarg{likelihood}{type}{multinomial}}\par
\par\textbf{\commandlabsubarg{likelihood}{type}{normal}}\par
\par\textbf{\commandlabsubarg{likelihood}{type}{pseudo}}\par
\defComLab{model}{Define an object of type \emph{model}}\par\par
\defSub{start\_year} {Define the first year of the model, immediately following initialisation}
\defSub{final\_year} {Define the final year of the model, excluding years in the projection period}
\defSub{min\_age} {Minimum age of individuals in the population}
\defSub{max\_age} {Maximum age of individuals in the population}
\defSub{age\_plus} {Define the oldest age or extra length midpoint (plus group size) as a plus group}
\defSub{initialisation\_phase\_labels} {Define the labels of the phases of the initialisation}
\defSub{time\_steps} {Define the labels of the time steps, in the order that they are applied, to form the annual cycle}
\defSub{length\_bins} {}
\defSub{length\_plus} {Is the last bin a plus group}
\defSub{base\_layer\_label} {Label for the base layer}
\defSub{latitude\_bounds} {Latitude bounds for the spatial domain, should include lower and upper bound, so there should be rows + 1 values}
\defSub{longitude\_bounds} {Longitude bounds for the spatial domain, should include lower and upper bound, so there should be columns + 1 values}
\defSub{nrows} {number of rows in spatial domain}
\defSub{ncols} {number of columns in spatial domain}
\defSub{sexed} {Is sex an attribute of then agents?}
\defSub{growth\_process\_label} {Label for the growth process in the annual cycle}
\defSub{natural\_mortality\_process\_label} {Label for the natural mortality process in the annual cycle}
\defComLab{observation}{Define an object of type \emph{observation}}\par\par
\defSub{label} {Label}
\defSub{type} {Type of observation}
\par\textbf{\commandlabsubarg{observation}{type}{age\_length}}\par
\defSub{time\_step} {The label of time-step that the observation occurs in}
\defSub{years} {The years of the observed values}
\defSub{selectivities} {Labels of the selectivities}
\defSub{cell\_layer} {The layer that indicates what area to summarise observations over.}
\defSub{cells} {The cells we want to generate observations for from the layer of cells supplied}
\defSub{number\_of\_samples} {The number of samples to collect from each cell}
\defSub{simulation\_likelihood} {Simulation likelihood to use}
\par\textbf{\commandlabsubarg{observation}{type}{biomass}}\par
\defSub{catchability} {The Catchability multiplier}
\defSub{years} {The years of the observed values}
\defSub{error\_value} {The error values of the observed values (note the units depend on the likelihood}
\defSub{selectivities} {Labels of the selectivities}
\defSub{proportion\_through\_mortality\_block} {Proportion through the mortality block of the time step to infer observation with}
\defSub{cell\_layer} {The layer that indicates what area to summarise observations over.}
\defSub{cells} {The cells we want to generate observations for from the layer of cells supplied}
\defSub{simulation\_likelihood} {Simulation likelihood to use}
\defSub{abundance} {Is the index biomass (abundance = false) or abundance (abundance = true}
\par\textbf{\commandlabsubarg{observation}{type}{mortality\_event\_biomass\_age\_clusters}}\par
\defSub{years} {The years of the observed values}
\defSub{ageing\_error} {Label of ageing error to use}
\defSub{process\_label} {Label of of removal process}
\defSub{fishery\_label} {Label of of removal process}
\defSub{cluster\_cv} {CV for randomly selecting clusters}
\defSub{composition\_type} {}
\defSub{cluster\_sigma} {Standard deviation for the M-H proposal distribution}
\defSub{age\_samples\_per\_cluster} {Number of age samples available to be aged per cluster}
\defSub{stratum\_weighting\_method} {Method to weight stratum estimates by}
\defSub{sexed} {You can ask to 'ignore' sex (only option for unsexed model), or generate composition for a particular sex, either 'male' or 'female}
\defSub{layer\_of\_stratum\_definitions} {The layer that indicates what the stratum boundaries are.}
\defSub{strata\_to\_include} {The cells which represent individual stratum to be included in the analysis, default is all cells are used from the layer}
\par\textbf{\commandlabsubarg{observation}{type}{mortality\_event\_biomass\_clusters}}\par
\defSub{years} {The years of the observed values}
\defSub{ageing\_error} {Label of ageing error to use}
\defSub{process\_label} {Label of of removal process}
\defSub{fishery\_label} {Label of of removal process}
\defSub{min\_age} {Minimum age}
\defSub{max\_age} {Maximum age}
\defSub{cluster\_cv} {CV for randomly selecting clusters}
\defSub{cluster\_sigma} {Standard deviation for the M-H proposal distribution}
\defSub{age\_samples\_per\_cluster} {Number of age samples available to be aged per cluster}
\defSub{length\_samples\_per\_cluster} {Number of age samples available to be aged per cluster}
\defSub{final\_age\_protocol} {What method do you want to use to calculate final age composition}
\defSub{age\_allocation\_method} {The method used to allocate aged individuals across the length distribution}
\defSub{stratum\_weighting\_method} {Method to weight stratum estimates by}
\defSub{sex} {You can ask to 'ignore' sex (only option for unsexed model), or generate composition for a particular sex, either 'male' or 'female}
\defSub{layer\_of\_stratum\_definitions} {The layer that indicates what the stratum boundaries are.}
\defSub{strata\_to\_include} {The cells which represent individual stratum to be included in the analysis, default is all cells are used from the layer}
\par\textbf{\commandlabsubarg{observation}{type}{mortality\_event\_biomass\_clusters\_original}}\par
\defSub{years} {The years of the observed values}
\defSub{ageing\_error} {Label of ageing error to use}
\defSub{process\_label} {Label of of removal process}
\defSub{fishery\_label} {Label of of removal process}
\defSub{cluster\_cv} {CV for randomly selecting clusters}
\defSub{cluster\_attribute} {What attribute do you want to link clusters by, either age or length}
\defSub{cluster\_correlation\_lambda} {The probability of being associated to a cluster based on distrance from attribute}
\defSub{age\_samples\_per\_cluster} {Number of age samples available to be aged per cluster}
\defSub{length\_samples\_per\_cluster} {Number of age samples available to be aged per cluster}
\defSub{minimum\_cluster\_weight\_to\_sample} {The minimum weight (tonnes) threshold to consider sampling, should be well in the distribution of cluster sizes}
\defSub{final\_age\_protocol} {What method do you want to use to calculate final age composition}
\defSub{age\_allocation\_method} {The method used to allocate aged individuals across the length distribution}
\defSub{sex} {You can ask to 'ignore' sex (only option for unsexed model), or generate composition for a particular sex, either 'male' or 'female}
\defSub{layer\_of\_stratum\_definitions} {The layer that indicates what the stratum boundaries are.}
\defSub{strata\_to\_include} {The cells which represent individual stratum to be included in the analysis, default is all cells are used from the layer}
\par\textbf{\commandlabsubarg{observation}{type}{mortality\_event\_biomass\_clusters\_second\_cut}}\par
\defSub{years} {The years of the observed values}
\defSub{ageing\_error} {Label of ageing error to use}
\defSub{process\_label} {Label of of removal process}
\defSub{fishery\_label} {Label of of removal process}
\defSub{min\_age} {Minimum age}
\defSub{max\_age} {Maximum age}
\defSub{cluster\_cv} {CV for randomly selecting clusters}
\defSub{cluster\_attribute} {What attribute do you want to link clusters by, either age or length}
\defSub{cluster\_correlation\_lambda} {The probability of being associated to a cluster based on distrance from attribute}
\defSub{age\_samples\_per\_cluster} {Number of age samples available to be aged per cluster}
\defSub{length\_samples\_per\_cluster} {Number of age samples available to be aged per cluster}
\defSub{minimum\_cluster\_weight\_to\_sample} {The minimum weight (tonnes) threshold to consider sampling, should be well in the distribution of cluster sizes}
\defSub{final\_age\_protocol} {What method do you want to use to calculate final age composition}
\defSub{age\_allocation\_method} {The method used to allocate aged individuals across the length distribution}
\defSub{sex} {You can ask to 'ignore' sex (only option for unsexed model), or generate composition for a particular sex, either 'male' or 'female}
\defSub{layer\_of\_stratum\_definitions} {The layer that indicates what the stratum boundaries are.}
\defSub{strata\_to\_include} {The cells which represent individual stratum to be included in the analysis, default is all cells are used from the layer}
\par\textbf{\commandlabsubarg{observation}{type}{mortality\_event\_biomass\_length\_clusters}}\par
\defSub{years} {The years of the observed values}
\defSub{process\_label} {Label of of removal process}
\defSub{fishery\_label} {Label of of removal process}
\defSub{cluster\_cv} {CV for randomly selecting clusters}
\defSub{composition\_type} {}
\defSub{cluster\_sigma} {Standard deviation for the M-H proposal distribution}
\defSub{length\_samples\_per\_cluster} {Number of age samples available to be aged per cluster}
\defSub{stratum\_weighting\_method} {Method to weight stratum estimates by}
\defSub{sexed} {You can ask to 'ignore' sex (only option for unsexed model), or generate composition for a particular sex, either 'male' or 'female}
\defSub{layer\_of\_stratum\_definitions} {The layer that indicates what the stratum boundaries are.}
\defSub{strata\_to\_include} {The cells which represent individual stratum to be included in the analysis, default is all cells are used from the layer}
\par\textbf{\commandlabsubarg{observation}{type}{mortality\_event\_biomass\_scaled\_age\_frequency}}\par
\defSub{years} {The years of the observed values}
\defSub{ageing\_error} {Label of ageing error to use}
\defSub{process\_label} {Label of of removal process}
\defSub{fishery\_label} {Label of of removal process}
\defSub{age\_allocation\_method} {The method used to allocate aged individuals across the length distribution}
\defSub{sex} {You can ask to 'ignore' sex (only option for unsexed model), or generate composition for a particular sex, either 'male' or 'female}
\defSub{layer\_of\_stratum\_definitions} {The layer that indicates what the stratum boundaries are.}
\defSub{strata\_to\_include} {The cells which represent individual stratum to be included in the analysis, default is all cells are used from the layer}
\par\textbf{\commandlabsubarg{observation}{type}{mortality\_event\_composition}}\par
\defSub{years} {The years of the observed values}
\defSub{ageing\_error} {Label of ageing error to use}
\defSub{process\_label} {Label of of removal process}
\defSub{fishery\_label} {Label of of removal process}
\defSub{stratum\_weighting\_method} {Method to weight stratum estimates by}
\defSub{sexed} {You can ask to 'ignore' sex (only option for unsexed model), or generate composition for a particular sex, either 'male' or 'female}
\defSub{composition\_type} {Is the composition Age or Length}
\defSub{normalise} {Are the compositions normalised to sum to one}
\defSub{simulation\_likelihood} {Simulation likelihood to use}
\defSub{min\_age} {Minimum age}
\defSub{max\_age} {Maximum age}
\defSub{plus\_group} {max age is a plus group}
\defSub{layer\_of\_stratum\_definitions} {The layer that indicates what the stratum boundaries are.}
\defSub{strata\_to\_include} {The cells which represent individual stratum to be included in the analysis, default is all cells are used from the layer}
\par\textbf{\commandlabsubarg{observation}{type}{mortality\_scaled\_age\_frequency}}\par
\defSub{years} {The years of the observed values}
\defSub{ageing\_error} {Label of ageing error to use}
\defSub{process\_label} {Label of of removal process}
\defSub{age\_allocation\_method} {The method used to allocate aged individuals across the length distribution}
\defSub{layer\_of\_stratum\_definitions} {The layer that indicates what the stratum boundaries are.}
\defSub{strata\_to\_include} {The cells which represent individual stratum to be included in the analysis, default is all cells are used from the layer}
\par\textbf{\commandlabsubarg{observation}{type}{process\_removals\_by\_age}}\par
\defSub{min\_age} {Minimum age}
\defSub{max\_age} {Maximum age}
\defSub{plus\_group} {Use age plus group}
\defSub{years} {Years for which there are observations}
\defSub{ageing\_error} {Label of ageing error to use}
\defSub{process\_label} {Label of of removal process}
\defSub{normalise} {Are the compositions normalised to sum to one}
\defSub{cell\_layer} {The layer that indicates what area to summarise observations over.}
\defSub{cells} {The cells we want to generate observations for from the layer of cells supplied}
\defSub{sex} {You can ask to 'ignore' sex (only option for unsexed model), or generate composition for a particular sex, either 'male' or 'female}
\defSub{simulation\_likelihood} {Simulation likelihood to use}
\par\textbf{\commandlabsubarg{observation}{type}{process\_removals\_by\_length}}\par
\defSub{years} {Years for which there are observations}
\defSub{process\_label} {Label of of removal process}
\defSub{cell\_layer} {The layer that indicates what area to summarise observations over.}
\defSub{cells} {The cells we want to generate observations for from the layer of cells supplied}
\defSub{simulation\_likelihood} {Simulation likelihood to use}
\par\textbf{\commandlabsubarg{observation}{type}{proportions\_at\_age}}\par
\defSub{min\_age} {Minimum age}
\defSub{max\_age} {Maximum age}
\defSub{sexed} {Observation split by sex}
\defSub{normalise} {Are the compositions normalised to sum to one}
\defSub{selectivities} {Labels of the selectivities}
\defSub{proportion\_through\_mortality\_block} {Proportion through the mortality block of the time step to infer observation with}
\defSub{plus\_group} {max age is a plus group}
\defSub{years} {Years for which to calculate an observation}
\defSub{ageing\_error} {Label of ageing error to use}
\defSub{cell\_layer} {The layer that indicates what area to summarise observations over.}
\defSub{cells} {The cells we want to generate observations for from the layer of cells supplied}
\defSub{time\_step} {The label of time-step that the observation occurs in}
\defSub{simulation\_likelihood} {Simulation likelihood to use}
\par\textbf{\commandlabsubarg{observation}{type}{tag\_recapture\_by\_age}}\par
\defSub{years} {The years of the observed values}
\defSub{tag\_release\_year} {The years that the tagged fish were released}
\defSub{ageing\_error} {Label of ageing error to use}
\defSub{process\_label} {Label of of removal process}
\defSub{release\_stratum} {Stratum that individuals were released in}
\defSub{recapture\_stratum} {Stratum to record recaptures, that were released in this year and in the release stratum}
\par\textbf{\commandlabsubarg{observation}{type}{tag\_recapture\_by\_length}}\par
\defSub{years} {The years of the observed values}
\defSub{tag\_release\_year} {The years that the tagged fish were released}
\defSub{sexed} {Seperate observation by sex}
\defSub{process\_label} {Label of of removal process}
\defSub{layer\_of\_strata\_definitions} {The layer that indicates what the stratum boundaries are.}
\defSub{release\_stratum} {Stratum that individuals were released in}
\defSub{recapture\_stratum} {Stratum to record recaptures, that were released in this year and in the release stratum}
\defComLab{process}{Define an object of type \emph{process}}\par\par
\defSub{label} {The label of the process}
\defSub{type} {The type of process}
\par\textbf{\commandlabsubarg{process}{type}{growth\_schnute\_with\_basic}}\par
\defSub{distribution} {the distribution to allocate the parameters to the agents}
\defSub{update\_growth\_parameters} {If an agent/individual moves do you want it to take on the growth parameters of the new spatial cell}
\defSub{cv} {The cv of the distribution}
\defSub{alpha\_layer\_label} {Label for the numeric layer that describes mean L inf by area}
\defSub{beta\_layer\_label} {Label for the numeric layer that describes mean k by area}
\defSub{t0\_layer\_label} {Label for the numeric layer that describes mean t0 by area}
\defSub{a\_layer\_label} {Label for the numeric layer that describes mean a in the weight calcualtion through space}
\defSub{b\_layer\_label} {Label for the numeric layer that describes mean b in the weight calcualtion through space}
\defSub{t0} {The value for t0 default = 0}
\defSub{alpha} {alpha value for schnute growth curve}
\defSub{beta} {beta value for schnute growth curve}
\defSub{a} {alpha value for weight at length function}
\defSub{b} {beta value for weight at length function}
\defSub{tau1} {reference age for y1}
\defSub{tau2} {reference age for y2}
\defSub{y1} {mean size at reference ages tau1}
\defSub{y2} {mean size at reference ages tau2}
\defSub{alpha} {}
\defSub{beta} {}
\defSub{tau1} {}
\defSub{tau2} {}
\par\textbf{\commandlabsubarg{process}{type}{growth\_von\_bertalanffy\_with\_basic}}\par
\defSub{distribution} {the distribution to allocate the parameters to the agents}
\defSub{update\_growth\_parameters} {If an agent/individual moves do you want it to take on the growth parameters of the new spatial cell}
\defSub{cv} {The cv of the distribution}
\defSub{linf\_layer\_label} {Label for the numeric layer that describes mean L inf by area}
\defSub{k\_layer\_label} {Label for the numeric layer that describes mean k by area}
\defSub{t0\_layer\_label} {Label for the numeric layer that describes mean t0 by area}
\defSub{t0} {The value for t0 default = 0}
\defSub{a\_layer\_label} {Label for the numeric layer that describes mean a in the weight calcualtion through space}
\defSub{b\_layer\_label} {Label for the numeric layer that describes mean b in the weight calcualtion through space}
\defSub{linf} {Value of mean L inf multiplied by the layer value if supplied}
\defSub{k} {Value of mean k multiplied by the layer value if supplied}
\defSub{a} {alpha value for weight at length function}
\defSub{b} {beta value for weight at length function}
\defSub{linf} {}
\defSub{k} {}
\par\textbf{\commandlabsubarg{process}{type}{maturity}}\par
\par\textbf{\commandlabsubarg{process}{type}{mortality\_baranov}}\par
\defSub{years} {years to apply the fishery process in}
\defSub{fishing\_mortality\_layers} {Spatial layer describing catch by cell for each year, there is a one to one link with the year specified, so make sure the order is right}
\defSub{minimum\_legel\_length} {The minimum legal length for this fishery, any individual less than this will be returned using some discard mortality}
\defSub{handling\_mortality} {if discarded due to being under the minimum legal length, what is the probability the individual will die when released}
\defSub{print\_extra\_info} {if you have process report for this process you can control the amount of information printed to the file.}
\defSub{scanning\_proportion\_of\_catch} {The proportion of catch scanned in each year}
\defSub{cv} {The cv of the distribution for the M distribution}
\defSub{m\_multiplier\_layer\_label} {Label for the numeric layer that describes a multiplier of M through space}
\defSub{natural\_mortality\_selectivity\_label} {Selectivity label for the natural mortality process}
\defSub{m} {Natural mortality for the model}
\par\textbf{\commandlabsubarg{process}{type}{mortality\_constant\_rate}}\par
\defSub{distribution} {the distribution to allocate the parameters to the agents}
\defSub{cv} {The cv of the distribution}
\defSub{m\_multiplier\_layer\_label} {Label for the numeric layer that describes a multiplier of M through space}
\defSub{m} {Natural mortality for the model}
\defSub{update\_mortality\_parameters} {If an agent/individual moves do you want it to take on the natural mortality parameters of the new spatial cell}
\defSub{time\_step\_ratio} {Time step ratios for the mortality rates to apply in each time step. See manual for how this is applied}
\defSub{m} {}
\par\textbf{\commandlabsubarg{process}{type}{mortality\_effort\_based}}\par
\defSub{selectivity} {Selectivity label}
\defSub{minimiser} {Label of the minimser to solve the problem}
\defSub{years} {years to apply the process}
\defSub{catches} {Total catch by year}
\defSub{effort\_values} {A vector of effort values, one for each enabled cell of the model, these should represent the variability of effort of the fishery mimicking}
\defSub{effort\_layer\_label} {A layer label that is a numeric layer label that contains effort values for each year.}
\defSub{starting\_value\_for\_lambda} {Total catch by year}
\par\textbf{\commandlabsubarg{process}{type}{mortality\_event\_biomass}}\par
\defSub{print\_census\_info} {if you have process report for this process you can control the amount of information printed to the file.}
\defSub{print\_tag\_recapture\_info} {if you have process report for this process you can control the amount of information printed to the file.}
\par\textbf{\commandlabsubarg{process}{type}{movement\_box\_transfer}}\par
\defSub{origin\_cell} {The origin cell associated with each spatial layer (should have a one to one relationship with specified layers), format follows row-col (1-2}
\defSub{probability\_layers} {Spatial layers (one layer for each origin cell) describing the probability of moving from an origin cell to all other cells in the spatial domain.}
\defSub{movement\_type} {What type of movement are you applying?}
\defSub{selectivity\_label} {Label for the selectivity block}
\par\textbf{\commandlabsubarg{process}{type}{movement\_preference}}\par
\defSub{d\_max} {The maxiumu diffusion value}
\defSub{time\_intervals} {The time interval that this movement process occurs over (k) in the documentation formula's}
\defSub{zeta} {An arbiituary parameter that controls curvature between preference and diffusion}
\defSub{brownian\_motion} {Just apply random movement, undirected movement}
\defSub{preference\_functions} {The preference functions to apply}
\defSub{preference\_layers} {The preference functions to apply}
\defSub{selectivity\_label} {Label for the selectivity block}
\par\textbf{\commandlabsubarg{process}{type}{nop}}\par
\par\textbf{\commandlabsubarg{process}{type}{recruitment\_beverton\_holt}}\par
\defSub{b0} {B0}
\defSub{recruitment\_layer\_label} {A label for the recruitment layer, that describes spatial distribution of recruits.}
\defSub{ssb} {A label for the SSB derived quantity}
\defSub{proportion\_male} {Proportion of recruits male}
\defSub{b0} {}
\defSub{proportion\_male} {}
\defSub{steepness} {Steepness}
\defSub{ycs\_values} {YCS (Year-Class-Strength) Values}
\par\textbf{\commandlabsubarg{process}{type}{recruitment\_constant}}\par
\defSub{b0} {B0}
\defSub{recruitment\_layer\_label} {A label for the recruitment layer, that describes spatial distribution of recruits.}
\defSub{ssb} {A label for the SSB derived quantity}
\defSub{proportion\_male} {Proportion of recruits male}
\defSub{b0} {}
\defSub{proportion\_male} {}
\par\textbf{\commandlabsubarg{process}{type}{tag\_shedding}}\par
\defSub{selectivity\_label} {Label for the selectivity block}
\defSub{time\_step\_ratio} {Time step ratios for the Shedding rates to apply in each time step. See manual for how this is applied}
\defSub{shedding\_rates} {Shedding rate per Tag release event}
\defSub{tag\_release\_region} {The Release region for the corresponding shedding rate, in the format 1-1 for the first row and first column, and '5-2' for the fifth row and secnd column}
\defSub{tag\_release\_year} {The Release Year for the corresponding shedding rate}
\defSub{years} {Years to execute the tag shedding process}
\par\textbf{\commandlabsubarg{process}{type}{tagging}}\par
\defSub{tag\_release\_layer} {Spatial layer describing catch by cell for each year, there is a one to one link with the year specified, so make sure the order is right}
\defSub{selectivities} {selectivity used to capture agents}
\defSub{handling\_mortality} {What is the handling mortality assumed for tagged fish, sometimes called initial mortality}
\defSub{years} {Years to execute the transition in}
\defComLab{report}{Define an object of type \emph{report}}\par\par
\defSub{label} {The label for the report}
\defSub{type} {The type of report}
\defSub{file\_name} {The File Name if you want this report to be in a seperate file}
\defSub{write\_mode} {The write mode}
\par\textbf{\commandlabsubarg{report}{type}{age\_frequency\_by\_cell}}\par
\defSub{years} {Years}
\defSub{time\_step} {Time Step label}
\defSub{do\_length\_frequency} {Print the report as length frequency not age.}
\par\textbf{\commandlabsubarg{report}{type}{age\_length\_matrix\_by\_cell}}\par
\defSub{years} {Years}
\defSub{time\_step} {Time Step label}
\par\textbf{\commandlabsubarg{report}{type}{ageing\_error\_matrix}}\par
\defSub{ageing\_error} {Ageing Error label}
\par\textbf{\commandlabsubarg{report}{type}{derived\_quantity}}\par
\par\textbf{\commandlabsubarg{report}{type}{initialisation\_partition}}\par
\defSub{do\_length\_frequency} {Print the report as length frequency not age.}
\par\textbf{\commandlabsubarg{report}{type}{likelihood}}\par
\defSub{likelihood} {Likelihood label that is reported}
\par\textbf{\commandlabsubarg{report}{type}{model\_attributes}}\par
\par\textbf{\commandlabsubarg{report}{type}{numeric\_layer}}\par
\defSub{layer\_label} {The Numeric Layer label that is reported}
\defSub{years} {Years}
\defSub{time\_step} {Time Step label}
\par\textbf{\commandlabsubarg{report}{type}{observation}}\par
\defSub{observation} {Observation label}
\par\textbf{\commandlabsubarg{report}{type}{process}}\par
\defSub{process} {Process label that is reported}
\par\textbf{\commandlabsubarg{report}{type}{selectivity}}\par
\defSub{selectivity} {Selectivity name}
\par\textbf{\commandlabsubarg{report}{type}{simulated\_observation}}\par
\defSub{observation} {Observation label}
\defSub{cells} {Cells to aggregate the observaton over.}
\par\textbf{\commandlabsubarg{report}{type}{standard\_header}}\par
\par\textbf{\commandlabsubarg{report}{type}{summarise\_agents}}\par
\defSub{years} {Years}
\defSub{time\_step} {Time Step label}
\defSub{number\_of\_agents} {Number of agents to summarise}
\par\textbf{\commandlabsubarg{report}{type}{time\_varying}}\par
\par\textbf{\commandlabsubarg{report}{type}{world\_age\_frequency}}\par
\defSub{years} {Years}
\defSub{time\_step} {Time Step label}
\defComLab{selectivity}{Define an object of type \emph{selectivity}}\par\par
\defSub{label} {The label for this selectivity}
\defSub{type} {The type of selectivity}
\defSub{length\_based} {Is the selectivity length based}
\defSub{include\_age\_zero} {Include 0 aged fish in selectivity (more for comparing with population models that start modelling fish at age = 1}
\par\textbf{\commandlabsubarg{selectivity}{type}{all\_values}}\par
\defSub{v} {V}
\par\textbf{\commandlabsubarg{selectivity}{type}{all\_values\_bounded}}\par
\defSub{l} {L}
\defSub{h} {H}
\defSub{v} {V}
\par\textbf{\commandlabsubarg{selectivity}{type}{constant}}\par
\defSub{c} {C}
\par\textbf{\commandlabsubarg{selectivity}{type}{double\_exponential}}\par
\defSub{x0} {X0}
\defSub{x1} {X1}
\defSub{x2} {X2}
\defSub{y0} {Y0}
\defSub{y1} {Y1}
\defSub{y2} {Y2}
\defSub{alpha} {Alpha}
\par\textbf{\commandlabsubarg{selectivity}{type}{double\_normal}}\par
\defSub{mu} {Mu}
\defSub{sigma\_l} {Sigma L}
\defSub{sigma\_r} {Sigma R}
\defSub{alpha} {Alpha}
\par\textbf{\commandlabsubarg{selectivity}{type}{increasing}}\par
\defSub{l} {Low}
\defSub{h} {High}
\defSub{v} {V}
\defSub{alpha} {Alpha}
\par\textbf{\commandlabsubarg{selectivity}{type}{inverse\_logistic}}\par
\defSub{a50} {A50}
\defSub{ato95} {aTo95}
\defSub{alpha} {Alpha}
\par\textbf{\commandlabsubarg{selectivity}{type}{knife\_edge}}\par
\defSub{e} {Edge}
\defSub{alpha} {Alpha}
\par\textbf{\commandlabsubarg{selectivity}{type}{logistic}}\par
\defSub{a50} {A50}
\defSub{ato95} {Ato95}
\defSub{alpha} {Alpha}
\par\textbf{\commandlabsubarg{selectivity}{type}{logistic\_producing}}\par
\defSub{l} {Low}
\defSub{h} {High}
\defSub{a50} {A50}
\defSub{ato95} {Ato95}
\defSub{alpha} {Alpha}
\defComLab{timestep}{Define an object of type \emph{timestep}}\par\par
\defSub{label} {The label of the timestep}
\defSub{processes} {The labels of the processes for this time step in the order that they occur}
\defComLab{timevarying}{Define an object of type \emph{timevarying}}\par\par
\defSub{label} {The time-varying label}
\defSub{type} {The time-varying type}
\defSub{years} {Years in which to vary the values}
\defSub{parameter} {The name of the parameter to time vary}
\par\textbf{\commandlabsubarg{time\_varying}{type}{annual\_shift}}\par
\defSub{values} {}
\defSub{a} {}
\defSub{b} {}
\defSub{c} {}
\defSub{scaling\_years} {}
\par\textbf{\commandlabsubarg{time\_varying}{type}{constant}}\par
\defSub{values} {Value to assign to addressable}
\par\textbf{\commandlabsubarg{time\_varying}{type}{exogenous}}\par
\defSub{a} {Shift parameter}
\defSub{exogeneous\_variable} {Values of exogeneous variable for each year}
\par\textbf{\commandlabsubarg{time\_varying}{type}{linear}}\par
\defSub{slope} {The slope of the linear trend (additive unit per year}
\defSub{intercept} {The intercept of the linear trend value for the first year}
\par\textbf{\commandlabsubarg{time\_varying}{type}{random\_draw}}\par
\defSub{mean} {Mean}
\defSub{sigma} {Standard deviation}
\defSub{distribution} {distribution}
\par\textbf{\commandlabsubarg{time\_varying}{type}{random\_walk}}\par
\defSub{mean} {Mean}
\defSub{sigma} {Standard deviation}
\defSub{upper\_bound} {Upper bound for the random walk}
\defSub{upper\_bound} {Lower bound for the random walk}
\defSub{rho} {Auto Correlation parameter}
\defSub{distribution} {distribution}
