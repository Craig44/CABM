\section{Introduction\label{sec:Introduction}} 

\IBM\ is a command line program that implements a generalised agent-based model (ABM). An ABM is a model that represents a fish stock as a collection of agents. An agent is defined as one or more fish with homogenous characteristics, i.e. length, weight and sex. When an agent represents a single individual, the ABM becomes an individual-based model (IBM) \citep{grimm2013individual}. Given fish stocks consist of millions if not billions of individuals, ABMs are often more practical than IBMs due to computational limitations, i.e. it requires large amounts of memory to record and modify millions of agents. ABMs use functions to grow, move, create and kill agents over time, termed agent dynamics. When summaries are made over all agents, stock level quantities are observed. Simulating stocks with this high level of detail allows heterogeneity in key dynamics such as growth and mortality. This is an advantage of ABMs as this heterogeneity is often approximated in other operating models (OM).


\IBM\ was created to emulate a stock over a fine spatial resolution domain, apply realistic movement and replicate complex fishing processes. These were all attributes of an OM that are of interest for exploring stock assessment methods. 

\IBM\ was designed to be flexible with respect to specifying agent dynamics, spatial resolution, timing of agent dynamics, simulated observations and model outputs. It was designed for emulating fish populations, but can be modified to represent other populations. \IBM\ is a program I developed during my PhD and I hope that it will be useful to others. The core modelled entity is an agent. An agent can represent many individual fish with identical characteristics. 

\IBM\ may not be as efficient when compared to bespoke agent-based models in the literature. This is because generality and flexibility comes with a computational cost, but I have found the program to be fast enough for most tasks that I have used it for. The bonus in generality is it has readable input syntax and error handling making it faster to develop and run models (in Theory).

\IBM\ applies an annual cycle each model year for a user defined number of years. The annual cycle consists of user defined number of discrete time-steps. In each time-step users must specify the agent dynamics, simulated observations and model outputs. It can simulate many different user defined quantities, for example removals-at-length or -age from an anthropogenic or exploitation event (e.g. fishery or other human impact), scientific survey and other biomass indices, and mark-recapture data.

The real power of \IBM\ is its use in a closed loop management strategy evaluation (MSE). Recent code changes have allowed \IBM\ to run a model run and then once it reaches assessments years it will run an R script and wait for future catches before continuing to the next assessment year (see Section~\ref{sec:mse-section} for more details).

\subsection{\I{Where to get \IBM }}
\IBM\ source code is hosted on GitHub, and can be found at \url{https://github.com/Craig44/CABM}\index{github}.

Currently you can either compile the code using the \subcommand{BuildSystem}, or download a pre-compiled version from \url{https://github.com/Craig44/CABM/releases}. This repository is self contained with all the required third-party libraries and has been developed for ease of compilation (it is very easy see the GitHub page for information). 

\subsection{\I{System requirements}}
\IBM\ is available for most IBM compatible machines running 64-bit \I{Linux} and \I{Microsoft Windows} operating systems, unfortunately \I{Mac} has not been tested.

Several of \IBM 's tasks are highly computer intensive and a fast processor is recommended. Depending on the model implemented, some of the \IBM\ tasks can take a considerable amount of processing time.

The program itself can require anywhere from a few gigs to 10's of gigs of hard-disk space, depending on the size and complexity of the model. Output files can also consume large amounts of disk space\index{Disk space}. Depending on the number and type of user output requests, the output could range from a few hundred kilobytes to several hundred megabytes. Several hundred megabytes of RAM may be required, depending on the spatial size of the model, number of agents, and complexity of processes and observations. For extremely large models, several gigabytes of RAM may occasionally be required. 

\subsection{\I{Necessary files}}

For both 64-bit Linux and Microsoft Windows, only the binary executable \texttt{cabm} or \texttt{cabm.exe} is required to run \IBM .


\IBM\ comes with an \href{http://www.r-project.org}{\R}\ \citep{R} package to assist in the post processing of \IBM\ output. We provide the \texttt{ibm} \R\ package for importing the \IBM\ output into \R\ (see Section \ref{sec:post-processing}).

\subsection{Getting help\index{Getting help}\index{User assistance}\index{Notifying errors}}

\IBM\ is distributed as 'officially' unsupported software although I am always happy to help any user who wants to give this a crack. The Development Team would appreciate being notified of any problems or errors in \IBM , please use the github page to post issues, see Section \ref{sec:reporting-errors} for the recommended template for reporting issues.

\subsection{Technical details\index{Technical specifications}}\label{sec:tech}

\IBM\ was compiled on Linux using \texttt{gcc} (\url{http://gcc.gnu.org}), the C/C++ compiler developed by the GNU Project (\url{http://gcc.gnu.org}). \textbf{note} this program uses OpenMP which is an option that you should tick if you want to compile the code. The 64-bit Linux \index{Linux} version was compiled using \texttt{gcc} version 5.1.0 20151010 Ubuntu Linux (\url{http://www.ubuntu.com/}). The Microsoft Windows (\url{http://www.microsoft.com})\index{Microsoft Windows} version was compiled using MingW (\url{http://www.mingw.org})\index{Mingw} \texttt{gcc} (tdm64-1) 5.1.0 (\url{http://gcc.gnu.org})\index{gcc}. The Microsoft Windows(\url{http://www.microsoft.com}) installer was built using the Inno Setup 5 (\url{http://www.jrsoftware.org/isdl.php}).

The random number generator\index{Random number generator} used by \IBM\ uses an implementation of the Mersenne twister random number generator \citep{796}. This, the command line functionality, matrix operations, and a number of other functions use the \href{http://www.boost.org/}{BOOST} C++ library (Version 1.58.0)\index{BOOST C++ library}. The threading capabilities are done using \href{https://www.openmp.org/}[OpenMP library]


\subsection{\I{The future for \IBM}}


\begin{itemize}
	\item Make it faster...
	\item \sout{Add multiple fisheries at the same time, often we have multiple fleets fishing e.g trawler and long-liners, currently we can have multiple fishing events but they do not occur simultaneously. This would have issues on data generation because the order of operations would be important and would get a bit fidgety.}
	\item \sout{test on a high performance computer (HPC) to see the capabilities if a user has access to this. how spatial can we go if we have threads of cores available, it would be cool to have 1000x1000 model. or alternatively use the threads in another space such as run multiple species in parallel and synchronise threads when they interact.}. I have explored multi-threading, but found unless the model has 1000's of  cells the overhead and complexity in code has not been worth continuing. 
	\item see it used in practice other than by my self, I am hoping that this could be used as a management strategy evaluation (MSE) tool. This has the ability to test management actions on a large scale such as test structural uncertainty in our models for MSE. Include the possible effects of climate change in the future etc. Watch this space.
\end{itemize}



