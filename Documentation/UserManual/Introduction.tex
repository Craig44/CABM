\section{Introduction\label{sec:Introduction}} 

\IBM\ ia a project I am currently working on during my PhD and I hope that it will be useful to some one out in the wide world. \IBM\ simulates a generalised individual based model that allows a great deal of choice in specifying the agent dynamics, spatial resolution, timing of events and model outputs. \IBM\ is designed for flexibility and is spatially explicit. The term Individual in this document also means agent or super individual, where a modelled entity can represent many entities with identical characteristics. Unfortunately I have used these terms interchangeably through out this document and the code. I am hoping to tidy this up for consistency in the future, apologies if this adds confusion.

\IBM\ may not be as fast as specific Individual based models as generality usually comes at a cost of speed, but I hope that is fast enough to do most tasks the users desire. The bonus in generality is if it gets uptake in the community you can have a little more faith in the underlying dynamics, also because of the syntax and error handling it should be easy to set up models than it would be to code your own (always pros and cons).

The time period and annual cycle of \IBM\ is completely defined by the user. It can simulate many different user defined quantities, for example removals-at-length or -age from an anthropogenic or exploitation event (e.g. fishery or other human impact), scientific survey and other biomass indices, and mark-recapture data.

The real power of \IBM\ I hope will be when it used in management strategy evaluation and population assessment model investigation.

\subsection{\I{Where to get \IBM }}
\IBM\ source code is hosted on github, and can be found at \url{https://github.com/Craig44/IBM}\index{github}.

Currently you have to compile the code, to get an executable but, the repository contains all the required thirdparty libraries and has been developed for ease of compilation (it is very easy see the github page for information). I am hoping for a beta release soon if you are interested in the state of the project I have set up a project workflow for the first release that can be found \IBM\ source code is hosted on github, and can be found at \url{https://github.com/Craig44/IBM/projects/1}\index{project reference}.

\subsection{\I{System requirements}}
\IBM\ is available for most IBM compatible machines running 64-bit \I{Linux} and \I{Microsoft Windows} operating systems.

Several of \IBM 's tasks are highly computer intensive and a fast processor is recommended. Depending on the model implemented, some of the \IBM\ tasks can take a considerable amount of processing time.

The program itself can require anywhere from a few gigs to 10's of gigs of hard-disk space, depending on the size and complexity of the model. Output files can also consume large amounts of disk space\index{Disk space}. Depending on the number and type of user output requests, the output could range from a few hundred kilobytes to several hundred megabytes. Several hundred megabytes of RAM may be required, depending on the spatial size of the model, number of agents, and complexity of processes and observations. For extremely large models, several gigabytes of RAM may occasionally be required. 

\subsection{\I{Necessary files}}

For both 64-bit Linux and Microsoft Windows, only the binary executable \texttt{ibm} or \texttt{ibm.exe} is required to run \IBM . No other software is required. We do not provide a version for 32-bit operating systems. 

\IBM\ comes with an \href{http://www.r-project.org}{\R}\ \citep{R} package to assist in the post processing of \IBM\ output. We provide the \texttt{ibm} \R\ package for importing the \IBM\ output into \R\ (see Section \ref{sec:post-processing}).

\subsection{Getting help\index{Getting help}\index{User assistance}\index{Notifying errors}}

\IBM\ is distributed as 'officially' unsupported software although I am always happy to help any user who wants to give this a crack. The Development Team would appreciate being notified of any problems or errors in \IBM , please use the github page to post issues, see Section \ref{sec:reporting-errors} for the recommended template for reporting issues.

\subsection{Technical details\index{Technical specifications}}\label{sec:tech}

\IBM\ was compiled on Linux using \texttt{gcc} (\url{http://gcc.gnu.org}), the C/C++ compiler developed by the GNU Project (\url{http://gcc.gnu.org}). \textbf{note} this program uses OpenMP which is an option that you should tick if you want to compile the code. The 64-bit Linux \index{Linux} version was compiled using \texttt{gcc} version 5.1.0 20151010 Ubuntu Linux (\url{http://www.ubuntu.com/}). The Microsoft Windows (\url{http://www.microsoft.com})\index{Microsoft Windows} version was compiled using MingW (\url{http://www.mingw.org})\index{Mingw} \texttt{gcc} (tdm64-1) 5.1.0 (\url{http://gcc.gnu.org})\index{gcc}. The Microsoft Windows(\url{http://www.microsoft.com}) installer was built using the Inno Setup 5 (\url{http://www.jrsoftware.org/isdl.php}).

The random number generator\index{Random number generator} used by \IBM\ uses an implementation of the Mersenne twister random number generator \citep{796}. This, the command line functionality, matrix operations, and a number of other functions use the \href{http://www.boost.org/}{BOOST} C++ library (Version 1.58.0)\index{BOOST C++ library}. The threading capabilities are done using \href{https://www.openmp.org/}[OpenMP library]


\subsection{\I{The future for \IBM}}
I still have 2.5 years of my PhD to go so I am hoping to maintain and continue development during that period. Things that are not in the Beta version that would be great to add one day (The dream-list)

\begin{itemize}
	\item Make it faster...
	\item Add multiple fisheries at the same time, often we have multiple fleets fishing e.g trawler and long-liners, currently we can have multiple fishing events but they do not occur simultaneously. This would have issues on data generation because the order of operations would be important and would get a bit nigly.
	\item test on a high performance computer (HPC) to see the capabilities if a user has access to this. how spatial can we go if we have threads of cores available, it would be cool to have 1000x1000 model. or alternatively use the threads in another space such as run multiple species in parallel and synchronise threads when they interact.	
	\item see it used in practice other than by my self, I am hoping that this could be used as a management strategy evaluation (MSE) tool. This has the ability to test management actions on a large scale such as test structural uncertainty in our models for MSE. Include the possible effects of climate change in the future etc
\end{itemize}



