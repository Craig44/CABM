\section{\I{Post processing output using \R} \label{sec:post-processing}}\index{Post processing}\index{Post processing section}

Hopefully when you get the bundle for this program there will be an \R\ package, this section describes how you can use that package to read in and view output from \IBM.





Comparing different initialisation starts, as mentioned in the Tips section~\ref{sec:tips} we highly recommend that you reduce the initialisation phase as much as possible. Below is some \R\ code that I often use when looking at the effects of different initialisation phase burn-ins.

\begin{lstlisting}
library(ibm) ## for extracting
library(ggplot2) ## for plotting
library(reshape2) ## for reshaping data so its ggplot friendly

## read in a reported output from a ibm-r runs
## ---------------------------------
## An important note before running the models below.
## if you do not include the following report in your configuration files
##
## @report init_2
## type initialisation_partition
##
## this code will not work
## ---------------------------------
ibm_30 = extract.run("output_30.log")
ibm_50 = extract.run("output_50.log")
ibm_80 = extract.run("output_80.log")
ibm_120 = extract.run("output_120.log")

names(ibm_50)
mat = ibm_50$init_2$`1`$values

temp = rbind(ibm_30$init_2$`1`$values,ibm_50$init_2$`1`$values, ibm_80$init_2$`1`$values, ibm_120$init_2$`1`$values)
temp$burnin =  c(rep("30", nrow(mat)),rep("50", nrow(mat)), rep("80", nrow(mat)), rep("120", nrow(mat)))

merged = melt(temp)
colnames(merged) = c("cell", "burnin", "age", "frequency")

## plot age frequency for each row of spatial gird
for (i in 1:5) {
	temp_data = merged[substring(merged$`cell`,0,1) == as.character(i),]
	p <- ggplot(temp_data, aes(x = age, y = frequency, color = burnin)) + geom_point()
	p + facet_grid(cell ~ ., scales="free_y")
	p
    Sys.sleep(2);
}
\end{lstlisting}



