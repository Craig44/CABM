
\section{\I{Syntax conventions, examples and niceties}\label{sec:examples}}

\subsection{Input File Specification}
The file format used for \IBM\ is based on the formats used for Casal2, CASAL and SPM. It's a standard text file that contains definitions organised into blocks.

Without exception, every object specified in a configuration file is part of a block. At the top level blocks have a one-to-one relationships with components in the system.


Some general notes about writing configuration files:
\begin{enumerate}
	\item Whitespace can be used freely. Tabs and spaces are both accepted
	\item A block ends only at the beginning of a new block or end of final configuration file
	\item You can include another configuration file from anywhere
	\item Included files are placed inline, so you can continue a block in a new file
	\item The configuration files support inline declarations of objects
\end{enumerate}

\subsubsection{Keywords And Reserved Characters}
In order to allow efficient creation of input files CASAL2's file format contains special keywords and characters that cannot be used for labels etc.

\paragraph*{\command Block Definitions}
Every new block in the configuration file must start with a block definition character. The reserved character for this is the \command character\\
Example:
{\small{\begin{verbatim}
@block1 <label>
type <type>

@block2 <label>
type <type>
\end{verbatim}}}

\paragraph*{'type' Keyword}
The 'type' keyword is used for declaring the sub-type of a defined block. Any block object that has multiple sub-types will use the type keyword.\\
Example:
{\small{\begin{verbatim}
@block1 <label>
type <sub_type>

@block2 <label>
type <sub_type>
\end{verbatim}}}

\paragraph*{\# (Single-Line Comment)}
Comments are supported in the configuration file in either single-line (to end-of-line) or multi-line\\
Example:
{\small{\begin{verbatim}
@block <label>
type <sub_type> #Descriptive comment
#parameter <value_1> – This whole line is commented out
parameter <value_1> #<value_2>(value_2 is commented out)
\end{verbatim}}}

\paragraph*{\commentstart\ \commentend\ (Multi-Line Comment)}
Multiple line comments are supported by surrounding the comments in \commentstart\ and \commentend\\
Example:
{\small{\begin{verbatim}
@block <label>
type <sub_type>
parameter <value_1>
parameter <value_1> <value_2>

\* 
	Do not load this process
	@block <label>
	type <sub_type>
	parameter <value_1>
	parameter <value_1> <value_2>
*\
\end{verbatim}}}

\paragraph*{$\{ \}$ (Indexing Parameters)}

Users can reference individual elements of a map using the \{ \} syntax, for example when estimating \subcommand{ycs\_values} you may only want to estimate a block of YCS not all of them say between 1975 and 2012.
Example:
{\small{\begin{verbatim}
		@estimate YCS
		parameter process[Recruitment].ycs_values{1975:2012}
		type uniform
		lower_bound
		upper_bound
		\end{verbatim}}}
	
\paragraph*{':' (Range Specifier)}
The range specifier allows you to specify a range of values at once instead of having to input them manually. Ranges can be either incremental or decremental.\\
Example:
{\small{\begin{verbatim}
@process my_recruitment_process
type constant_recruitment
years_to_run 1999:2009 #With range specifier

@process my_mortality_process
type natural_mortality
years_to_run 2000 2001 2002 2003 2004 2005 2006 2007 #Without range specifier
\end{verbatim}}}


\paragraph*{'table' and 'end\_table' Keyword}
The table keyword is used to define a table of information used as a parameter. The line following the table declaration must contain a list of columns to be used. Following lines are rows of the table. Each row must have the same number of values as the number of columns specified.
The table definition must end with the 'end\_table' keyword on it's own line.
The first row of a table will be the name of the columns if required.\\
Example:
{\small{\begin{verbatim}
@block <label>
type <sub_type>
parameter <value_1>
table <table_label>
<column_1> <column_2> <column_n>
<row1_value1> <row1_value2> <row1_valueN>
<row2_value1> <row2_value2> <row2_valueN>
end_table
\end{verbatim}}}

\paragraph*{[ ] (Inline Declarations)}
When an object takes the label of a target object as a parameter this can be replaced with an inline declaration. An inline declaration is a complete declaration of an object one 1 line. This is designed to allow the configuration writer to simplify the configuration writing process.\\
Example:
{\small{\begin{verbatim}
#With inline declaration with label specified for time step
@model
time_steps step_one=[type=iterative; processes=recruitment ageing]

#With inline declaration with default label (model.1)
@model
time_steps [type=iterative; processes=recruitment ageing]

#Without inline declaration
@model
time_steps step_one

@time_step step_one
processes recruitment ageing
\end{verbatim}}}


\paragraph*{\I{Parameters}\label{sec:params}}
\IBM\ also allows parameters that are of type vector or map to be referenced and estimated partially. An example of a parameter that is type vector is \texttt{ycs\_values} in a recruitment process. Let say a recruitment block was specified as follows,
{\small{\begin{verbatim}
		@process WestRecruitment
		type recruitment_beverton_holt
		r0 400000
		years
		ycs_values 1 1 1 1 1 1 1 1 
		ycs_years 1975:1983
		An alternative specification to the sequence of values you can use an astrix to
		shorthand repeating integers e.g.
		
		ycs_values 1*8
		
		steepness 0.9
		age 1
		\end{verbatim}}}

Lets say we wanted to only estimate the last four years of the parameter \texttt{process[WestRecruitment].ycs\_values}. This can be done as specified in the following \command{estimate} block,

{\small{\begin{verbatim}
		@estimate
		parameter process[WestRecruitment].ycs_values{1979:1983}
		type uniform
		lower_bound 0.1 0.1 0.1 0.1
		upper_bound  10  10  10  10 
		\end{verbatim}}}

Note the first element of a vector is indexed by 1. This syntax can be applied to parameters that are of type map as well, for information on what type a parameter is see the syntax section. An example of a parameter that is of type map is \command{time\_varying}\texttt{[label].type=constant}. For the following \command{time\_varying} block,

{\small{\begin{verbatim}
		@time_varying q_step1
		type constant
		parameter catchability[Fishq].q
		years 	1992	1993	1994	1995
		value 	0.2		0.2		0.2		0.2	
		\end{verbatim}}}

In this example a user may want to estimate only one element of the map (say 1992), but force all other years to be the same as the one estimate. This can be done in an estimate block as follows,
{\small{\begin{verbatim}
		@estimate
		parameter time_varying[q_step1].value{1992}
		same time_varying[q_step1].value{1993:1995}
		type uniform
		lower_bound 0.1 0.1 0.1 0.1
		upper_bound  10  10  10  10 
		\end{verbatim}}}
\paragraph*{\I{In line declaration}\label{sec:declare}}
In line declarations can help shorten models by passing \command{} blocks, for example 
{\small{\begin{verbatim}
		@observation chatCPUE
		type biomass
		catchability [q=6.52606e-005]
		time_step one
		categories male+female
		selectivities chatFselMale chatFselFemale
		likelihood lognormal
		years 1992:2001
		time_step_proportion 1.0
		obs 1.50 1.10 0.93 1.33 1.53 0.90 0.68 0.75 0.57 1.23
		error_value 0.35
		
		@estimate 
		parameter catchability[chatTANbiomass.one].q
		type uniform_log
		lower_bound 1e-2
		upper_bound 1
		In line declaration tips
		\end{verbatim}}}

In the above code we are defining and estimating catchability without explicitly creating an \command{catchability} block.

When you do an inline declaration the new object will be created with the name of the creator's \texttt{label.index} where index will be the word if it's one-nine and the number if it's 10+, for example,
{\small{\begin{verbatim}
		@mortality halfm
		selectivities [type=constant; c=1]
		
		would create
		@selectivity halfm.one
		\end{verbatim}}}

if there were 10 categories all with there own selectivity the $10^th$ selectivity would be labelled,

{\small{\begin{verbatim}
		@selectivity halfm.10
		\end{verbatim}}}


\subsection{Processes}
Processes are special in how they can be defined, all throughout this document we have been referring to specifying a process as follows,

{\small{\begin{verbatim}
		@process Recruitment
		type recruitment_beverton_holt
		\end{verbatim}}}
However for convenience and for file clarity you could equally specify this block as follows,
{\small{\begin{verbatim}
		@recruitment Recruitment
		type beverton_holt
		\end{verbatim}}}

The trick is that you can replace the keyword \subcommand{process} with the first word of the process type, in the example above this is the \subcommand{recruitment} this can be away of creating more reader friendly/lay term configuration scripts. More examples follow;

{\small{\begin{verbatim}
		@mortality Fishing_and_M
		type instantaneous
		\end{verbatim}}}

{\small{\begin{verbatim}
		@transition Migration
		type category
		\end{verbatim}}}
\textsl{}
