\section{Population command and subcommand syntax\label{sec:population-syntax}}

\subsection{\I{Model structure}}
\input{Syntax/Model} 

\subsection{\I{Initialisation}}
\input{Syntax/InitialisationPhase} 

\subsection{\I{Time-steps}}
\input{Syntax/TimeStep} 

\subsection{\I{Processes}}
\input{Syntax/Process} 

\subsection{\I{Time varying parameters}}
\input{Syntax/TimeVarying}

\subsection{\I{Derived quantities}}
\input{Syntax/DerivedQuantity} 

\subsection{\I{Selectivities}}
\input{Syntax/Selectivity} 

\subsection{\I{Layers}}
\input{Syntax/Layer} 

An example of how to specify a table in the layers for an Int-layer or integer layer you could specify

{\small{\begin{verbatim}
table layer
1 1 1 1
1 1 0 1
1 1 1 1
end_table
\end{verbatim}}}

for a Numeric layer you could have decimal and negative values
{\small{\begin{verbatim}
table layer
2.2 3.2 -23
6.4 -4.5 2.3
end_table
\end{verbatim}}}

\section{Observation command and subcommand syntax\label{sec:observation-syntax}}

%Note: this code auto generated by the Casal2 build system. 

\subsection{\I{Observation types}}

The observation types available are,

\begin{description}
  \item Observations of proportions of individuals by age class
  \item Observations of proportions of individuals between categories within each age class
  \item Relative and absolute abundance observations
  \item Relative and absolute biomass observations
\end{description}

Each type of observation requires a set of subcommands and arguments specific to that process.

\section{The observation section\label{sec:observation-section}}

\subsection{\I{Observations}\label{sec:Observations}\index{Observations}}


\subsection{\I{Likelihoods}}
\input{Syntax/Likelihood}

\section{Report command and subcommand syntax\label{sec:report-syntax}}
\subsection{\I{Report commands and subcommands}}

\section{\I{The report section}\label{sec:report-section}}\index{Reports}\index{Reports section}
The report section specifies the printouts and other outputs from the model. \IBM\ does not, in general, produce any output unless requested by a valid \command{report} block. 

Reports from \IBM\ can be defined to print partition and states objects at a particular point in time, observation summaries, estimated parameters and objective function values. See below for a more extensive list of report types, and an example of an observation report.



\section{Including commands from other files\label{sec:general-syntax}}

\defComArg{include}{file}{\I{Include an external file}}

\defArg{file}{The name of the external file to include}
\defType{string}
\defDefault{No default}
\defValue{A valid external file}
\defCondition{The file name must be enclosed in double quotes}
\defExample{!\texttt{include} \argument{\ "my\_file.ibm"}}
\defNote{!\texttt{include} does not denote the end of the previous command block as is the case for all other commands}
